\documentclass[11pt]{article}

\usepackage{fullpage,graphicx,latexsym,picinpar,amsbsy,amsmath,amsfonts}

\input{macros.tex}

\begin{document}
    \textbf{20240416}
    \textbf{Chapter 3 details}\\
    Prerequisites to begin working on a project:\\
    (0) - Starts with the Problem Definition (what details are desired)\\
    (a) List of requirements (gathered via user stories)\\
    (b) Architecture of the Problem\\
    

    \textit{CONCEPT}: The earlier a defect occurs in the process and later 
    it is detected, the more costly the problem\\
    \\
    \textbf{UML Usage}\\
    $\rightarrow$ Only class diagrams will be used for project\\
    \indent $\bullet$ Organize class hierarchy\\
    \indent $\bullet$ - sign is private. + sign is public.  \\
    \indent $\bullet$ Generalization relationship - inheritance - triangle to base class  \\
    \indent $\bullet$ Association relationship - aggregation - stored as a variable in another class - solid line  \\
    \indent\indent$\bullet$ Full or empty diamond included for composition versus aggregation \\
    \indent\indent$\bullet$ Composition is most typical; aggregation is different (more like working together)\\
    \indent $\bullet$ Object type is not relevant in UML (pointer or not) - only shows up within class card  \\
    \\ \\
    \textbf{20240417 Discussion : GDB and Valgrind}\\
    g++ filename.ext -g -o newfile.exe\\
    "(gdb) break <line-number>"\\
    (gdb) print <variablename>\\
    (gdb) step - goes into the function code\\
    (gdb) next - runs function but does not enter function code\\
    (gdp) continue - runs to end\\
    (gdb) info breakpoints\\
    (gdb) del break 1\\
    (gdb) quit\\
    exiting the debugger also removes breakpoints\\
\\
Valgrind:\\
Memory debugging via memcheck
g++ -g -O0 *.cpp -o newfile.exe\\
valgrind --leak-check=full filename.exe\\
--track-origins=yes gives locations of memory leaks\\
\\
\textit{additional valgrind details:}\\
valgrind ./filename.exe (runs valgrind and gives list of issues)\\
Commands show up in the output for further commands\\


    \end{document}