\documentclass[11pt]{article}

\usepackage{fullpage,graphicx,latexsym,picinpar,amsbsy,amsmath,amsfonts}

           

%%%%%%%%%%%%%%%%%%%%%%%%%%%%%%%%%%%%%%%%%%%%%%%%%%%%%%%%%%%%%%%%%%%%%%%%%%%%%%%%%%%
%%%%%%%%%%%  LETTERS 
%%%%%%%%%%%%%%%%%%%%%%%%%%%%%%%%%%%%%%%%%%%%%%%%%%%%%%%%%%%%%%%%%%%%%%%%%%%%%%%%%%%

\newcommand{\barx}{{\bar x}}
\newcommand{\bary}{{\bar y}}
\newcommand{\barz}{{\bar z}}
\newcommand{\bart}{{\bar t}}

\newcommand{\bfP}{{\bf{P}}}

%%%%%%%%%%%%%%%%%%%%%%%%%%%%%%%%%%%%%%%%%%%%%%%%%%%%%%%%%%%%%%%%%%%%%%%%%%%%%%%%%%%
%%%%%%%%%%%%%%%%%%%%%%%%%%%%%%%%%%%%%%%%%%%%%%%%%%%%%%%%%%%%%%%%%%%%%%%%%%%%%%%%%%%
                                                                                
\newcommand{\parend}[1]{{\left( #1  \right) }}
\newcommand{\spparend}[1]{{\left(\, #1  \,\right) }}
\newcommand{\angled}[1]{{\left\langle #1  \right\rangle }}
\newcommand{\brackd}[1]{{\left[ #1  \right] }}
\newcommand{\spbrackd}[1]{{\left[\, #1  \,\right] }}
\newcommand{\braced}[1]{{\left\{ #1  \right\} }}
\newcommand{\leftbraced}[1]{{\left\{ #1  \right. }}
\newcommand{\floor}[1]{{\left\lfloor #1\right\rfloor}}
\newcommand{\ceiling}[1]{{\left\lceil #1\right\rceil}}
\newcommand{\barred}[1]{{\left|#1\right|}}
\newcommand{\doublebarred}[1]{{\left|\left|#1\right|\right|}}
\newcommand{\spaced}[1]{{\, #1\, }}
\newcommand{\suchthat}{{\spaced{|}}}
\newcommand{\numof}{{\sharp}}
\newcommand{\assign}{{\,\leftarrow\,}}
\newcommand{\myaccept}{{\mbox{\tiny accept}}}
\newcommand{\myreject}{{\mbox{\tiny reject}}}
\newcommand{\blanksymbol}{{\sqcup}}
                                                                                                                         
\newcommand{\veps}{{\varepsilon}}
\newcommand{\Sigmastar}{{\Sigma^\ast}}
                           
\newcommand{\half}{\mbox{$\frac{1}{2}$}}    
\newcommand{\threehalfs}{\mbox{$\frac{3}{2}$}}   
\newcommand{\domino}[2]{\left[\frac{#1}{#2}\right]}  

%%%%%%%%%%%% complexity classes

\newcommand{\PP}{\mathbb{P}}
\newcommand{\NP}{\mathbb{NP}}
\newcommand{\PSPACE}{\mathbb{PSPACE}}
\newcommand{\coNP}{\textrm{co}\mathbb{NP}}
\newcommand{\DLOG}{\mathbb{L}}
\newcommand{\NLOG}{\mathbb{NL}}
\newcommand{\NL}{\mathbb{NL}}

%%%%%%%%%%% decision problems

\newcommand{\PCP}{\sc{PCP}}
\newcommand{\Path}{\sc{Path}}
\newcommand{\GenGeo}{\sc{Generalized Geography}}

\newcommand{\malytm}{{\mbox{\tiny TM}}}
\newcommand{\malycfg}{{\mbox{\tiny CFG}}}
\newcommand{\Atm}{\mbox{\rm A}_\malytm}
\newcommand{\complAtm}{{\overline{\mbox{\rm A}}}_\malytm}
\newcommand{\AllCFG}{{\mbox{\sc All}}_\malycfg}
\newcommand{\complAllCFG}{{\overline{\mbox{\sc All}}}_\malycfg}
\newcommand{\complL}{{\bar L}}
\newcommand{\TQBF}{\mbox{\sc TQBF}}
\newcommand{\SAT}{\mbox{\sc SAT}}

%%%%%%%%%%%%%%%%%%%%%%%%%%%%%%%%%%%%%%%%%%%%%%%%%%%%%%%%%%%%%%%%%%%%%%%%%%%%%%%%%%%
%%%%%%%%%%%%%%% for homeworks
%%%%%%%%%%%%%%%%%%%%%%%%%%%%%%%%%%%%%%%%%%%%%%%%%%%%%%%%%%%%%%%%%%%%%%%%%%%%%%%%%%%

\newcommand{\student}[2]{%
{\noindent\Large{ \emph{#1} SID {#2} } \hfill} \vskip 0.1in}

\newcommand{\assignment}[1]{\medskip\centerline{\large\bf CS 111 ASSIGNMENT {#1}}}

\newcommand{\duedate}[1]{{\centerline{due {#1}\medskip}}}     

\newcounter{problemnumber}                                                                                 

\newenvironment{problem}{{\vskip 0.1in \noindent
              \bf Problem~\addtocounter{problemnumber}{1}\arabic{problemnumber}:}}{}

\newcounter{solutionnumber}

\newenvironment{solution}{{\vskip 0.1in \noindent
             \bf Solution~\addtocounter{solutionnumber}{1}\arabic{solutionnumber}:}}
				{\ \newline\smallskip\lineacross\smallskip}

\newcommand{\lineacross}{\noindent\mbox{}\hrulefill\mbox{}}

\newcommand{\decproblem}[3]{%
\medskip
\noindent
\begin{list}{\hfill}{\setlength{\labelsep}{0in}
                       \setlength{\topsep}{0in}
                       \setlength{\partopsep}{0in}
                       \setlength{\leftmargin}{0in}
                       \setlength{\listparindent}{0in}
                       \setlength{\labelwidth}{0.5in}
                       \setlength{\itemindent}{0in}
                       \setlength{\itemsep}{0in}
                     }
\item{{{\sc{#1}}:}}
                \begin{list}{\hfill}{\setlength{\labelsep}{0.1in}
                       \setlength{\topsep}{0in}
                       \setlength{\partopsep}{0in}
                       \setlength{\leftmargin}{0.5in}
                       \setlength{\labelwidth}{0.5in}
                       \setlength{\listparindent}{0in}
                       \setlength{\itemindent}{0in}
                       \setlength{\itemsep}{0in}
                       }
                \item{{\em Instance:\ }}{#2}
                \item{{\em Query:\ }}{#3}
                \end{list}
\end{list}
\medskip
}

%%%%%%%%%%%%%%%%%%%%%%%%%%%%%%%%%%%%%%%%%%%%%%%%%%%%%%%%%%%%%%%%%%%%%%%%%%%%%%%%%%%
%%%%%%%%%%%%% for quizzes
%%%%%%%%%%%%%%%%%%%%%%%%%%%%%%%%%%%%%%%%%%%%%%%%%%%%%%%%%%%%%%%%%%%%%%%%%%%%%%%%%%%

\newcommand{\quizheader}{ {\large NAME: \hskip 3in SID:\hfill}
                                \newline\lineacross \medskip }


%%%%%%%%%%%%%%%%%%%%%%%%%%%%%%%%%%%%%%%%%%%%%%%%%%%%%%%%%%%%%%%%%%%%%%%%%%%%%%%%%%%
%%%%%%%%%%%%% for final
%%%%%%%%%%%%%%%%%%%%%%%%%%%%%%%%%%%%%%%%%%%%%%%%%%%%%%%%%%%%%%%%%%%%%%%%%%%%%%%%%%%

\newcommand{\namespace}{\noindent{\Large NAME: \hfill SID:\hskip 1.5in\ }\\\medskip\noindent\mbox{}\hrulefill\mbox{}}



\begin{document}
    \textbf{20240416}
    \textbf{Chapter 3 details}\\
    Prerequisites to begin working on a project:\\
    (0) - Starts with the Problem Definition (what details are desired)\\
    (a) List of requirements (gathered via user stories)\\
    (b) Architecture of the Problem\\
    

    \textit{CONCEPT}: The earlier a defect occurs in the process and later 
    it is detected, the more costly the problem\\
    \\
    \textbf{UML Usage (4/16 and 4/18)}\\
    $\rightarrow$ Only class diagrams will be used for project\\
    \indent $\bullet$ Organize class hierarchy\\
    \indent $\bullet$ - sign is private. + sign is public.  \\
    \indent $\bullet$ Generalization relationship - inheritance - triangle to base class  \\
    \indent $\bullet$ Association relationship - aggregation - stored as a variable in another class - solid line  \\
    \indent\indent$\bullet$ Full or empty diamond included for composition versus aggregation \\
    \indent\indent$\bullet$ Composition is most typical; aggregation is different (more like working together)\\
    \indent $\bullet$ Object type is not relevant in UML (pointer or not) - only shows up within class card  \\
    \indent $\bullet$ Book+Pages are composition; not a book without pages/cover \\
    \indent $\bullet$ Dependency (third relationship) \\
    \indent\indent $\bullet$ Example: function in class A uses class B in a function \\
    \indent $\bullet$ Place Abstract Classes with italic class name in UML \\
    \indent $\bullet$ Entity versus Boundary versus Control \\
    \indent\indent $\bullet$ MVC design - Model View Controller \\
    \indent\indent $\bullet$ Model: Entity\\
    \indent\indent $\bullet$ View: Boundary (user interface)\\
    \indent\indent $\bullet$ Controller: Control that manipulates interaction between model and view\\
    \indent\indent $\bullet$ Separation of UI and model code allows for simple classes \\
    \indent $\bullet$ 
    \\ \\
    \textbf{20240417 Discussion : GDB and Valgrind}\\
    g++ filename.ext -g -o newfile.exe\\
    (gdb) break <line-number>\\
    (gdb) print <variablename>\\
    (gdb) step - goes into the function code\\
    (gdb) next - runs function but does not enter function code\\
    (gdp) continue - runs to end\\
    (gdb) info breakpoints\\
    (gdb) del break 1\\
    (gdb) quit\\
    exiting the debugger also removes breakpoints\\
\\
Valgrind:\\
Memory debugging via memcheck
g++ -g -O0 *.cpp -o newfile.exe\\
valgrind --leak-check=full filename.exe\\
--track-origins=yes gives locations of memory leaks\\
\\
\textit{additional valgrind details:}\\
valgrind ./filename.exe (runs valgrind and gives list of issues)\\
Commands show up in the output for further commands\\



    \end{document}